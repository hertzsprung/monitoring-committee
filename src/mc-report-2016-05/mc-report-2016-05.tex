\documentclass[a4paper,11pt]{article}
\usepackage{fullpage}
\usepackage{standalone}
\usepackage[utf8]{inputenc}
\usepackage[british]{babel}
\usepackage{csquotes}
\usepackage[T1]{fontenc}
\usepackage{mathptmx}
\usepackage{natbib}
\usepackage[final,babel]{microtype}
\usepackage[hidelinks]{hyperref}
\usepackage{doi}
\usepackage{siunitx}
\usepackage[margin=3pt]{subcaption}
\usepackage{xcolor}
\PassOptionsToPackage{final}{graphicx}
\usepackage{tikz}
\usetikzlibrary{arrows}
\usetikzlibrary{patterns}
\usepackage{bm}
\usepackage{booktabs}
\usepackage{tabularx}

\title{Monitoring Committee Progress Report \#3\\
\vspace*{1em}
\Large{Numerical Representation of Mountains in Atmospheric Models}}
\author{James Shaw
\vspace{0.5em} \\
\large{Supervisors: Hilary Weller, John Methven, Terry Davies}
\vspace{0.5em} \\
\large{Monitoring Committee: Maarten Ambaum, Paul Williams}}
\date{31st May 2016}

\makeatletter
\AtBeginDocument{
  \hypersetup{
    pdftitle = {Monitoring Committee Progress Report \#3},
    pdfauthor = {James Shaw}
  }
}
\makeatother


\newcommand{\TODO}[1]{\textcolor{purple}{TODO: \emph{#1}}}
\begin{document}
\newcommand{\exner}{\Pi}
\maketitle

Atmospheric models solve the equations of motion in a discrete form on a three-dimensional mesh, with the lower boundary representing the terrain surface.
My PhD project seeks to make numerical weather and climate predictions more accurate by improving the mesh's representation of the terrain, and by improving the numerical schemes that operate on these meshes.

Most operational models apply a transformation to the vertical coordinate so that the mesh's lower boundary represents the terrain.  This terrain-following coordinate transformation reduces numerical accuracy in the calculation of horizontal pressure gradients \citep{gary1973,zaengl2012} and advection terms \citep{schaer2002}.  Inaccuracies are larger near steep terrain.  Since the basic terrain following coordinate was developed by \citet{galchen-somerville1975a}, there have been proposals to make coordinate transformations that are increasingly orthogonal \citep{simmons-burridge1981,schaer2002,leuenberger2010,klemp2011}.
The cut cell method is an alternative to terrain-following coordinates.  Cut cell meshes are orthogonal everywhere except at the surface so that pressure gradient errors and advection errors are reduced, especially near steep terrain \citep{lock2012,good2014}.
To illustrate the two different methods, a wave-shaped mountain is represented using a basic terrain following mesh (figure~\ref{fig:btf-mesh}) and a cut cell mesh (figure~\ref{fig:cutcell-mesh}).

\begin{figure}
	\centering
	\subcaptionbox{Basic terrain following mesh \label{fig:btf-mesh}}[.3\linewidth]{\TODO{btf mesh}}
	\subcaptionbox{Cut cell mesh \label{fig:cutcell-mesh}}[.3\linewidth]{\TODO{cut cell mesh}}
	\caption{\TODO{btf and cut cell meshes}}
	\label{fig:meshes}
\end{figure}


\section{Training}
\subsection*{Mathematics modules}
\begin{tabular}{l l l l}
Spring 2016	& MA3NAT & Numerical Analysis II & unassessed \\
Spring 2015	& MAMNSP & Numerical Solution of Partial Differential Equations  & 78\% \\
\end{tabular}

\subsection*{RRDP modules}
\begin{tabular}{l l}
24 Mar 2016	& Voice coaching: looking after your voice \\
26--27 Jan 2016 & Preparing to teach (introduction, marking \& feedback, leading small groups) \\
2 Dec 2015	& An essential guide to critical academic writing \\
17 Nov 2015	& Understanding the UK higher education context \\
19 May 2015	& How to avoid plagiarism \\
10 Mar 2015	& How to write a literature review \\
19 Feb 2015	& How to write a paper \\
\end{tabular}

\subsection*{External courses}
\begin{tabular}{l l}
June 2016 & Dynamical core intercomparison project summer school, NCAR \\
June 2015 & Advanced numerical methods for Earth-system modelling, ECMWF \\
\end{tabular}

\subsection*{Conferences and workshops}
\begin{tabularx}{\linewidth}{l l X}
October 2016 & Speaker & Numerical and computational methods for simulation of all-scale geophysical flows, ECMWF \\
July 2016 & Attendee & 1st GungHo Network meeting, Daresbury Laboratory \\
November 2015 & Attendee & GungHo workshop on next generation weather and climate prediction, UK Met Office \\
June 2015 & Attendee & Hoskins@70 \\
June 2015 & Poster & SCENARIO DTP conference \\
March 2015 & Speaker & Galerkin methods with applications in weather and climate forecasting, ICMS \\
\end{tabularx}

\subsection*{Presentations}
\begin{tabularx}{\linewidth}{l l X}
23 Mar 2016 & Quo Vadis & Numerical representation of orography in dynamical cores \\
17 Feb 2016 & PhD group & Multidimensional advection schemes for arbitrary meshes \\
9 Feb 2016 & Mesoscale group & Curl-free pressure gradients for accurate modelling of cold air pools \\
19 Oct 2015 & HHH group & Improving modelled mountain flows with alternative representations of terrain \\
27 Apr 2015 & HHH group & A like-for-like comparison between terrain following and cut cell grids \\
21 Apr 2015 & PhD group & Discrete vector calculus on Arakawa C grids \\
12 Feb 2015 & UK Met office & Poster presentation \\
18 Jan 2015 & PhD group & Python and linux tips \\
17 Dec 2014 & MPECDT jamboree & Poster presentation \\
12 Sep 2014 & Lunchtime seminar  & Gain control of your documents and code: hands-on with revision control and build automation \\
\end{tabularx}

\subsection*{Teaching}
\begin{tabular}{l l l}
Oct 2015 & Teaching assistant & MTMG02 atmospheric physics \\
Sep 2015 & Teaching assistant & NCAS summer school \\
Sep 2014 & Course teacher & MPE python and linux short course \\
\end{tabular}

\subsection*{Outreach}
\begin{tabular}{l l l}
July 2016 & Organising filming with scientist and videographer, Simon Clark \\
14 Jul 2015 & Schools physicist of the year awards \\
14 Jun 2015 & East Reading festival \\
15 Feb 2015 & Brighton science festival \\
\end{tabular}

\subsection*{Collaboration}
Yumeng, Leibniz Institute for Tropospheric Research, Mattijs Janssens, Christiana Skea 

Visiting NCAR, hosted by Ram Nair

\bibliographystyle{ametsoc2014}                                                 
\small{\bibliography{references}}

\end{document}
