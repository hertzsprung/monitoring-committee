\documentclass[a4paper,11pt]{article}
\usepackage{fullpage}
\usepackage{standalone}
\usepackage[utf8]{inputenc}
\usepackage[british]{babel}
\usepackage{csquotes}
\usepackage[T1]{fontenc}
\usepackage{amsmath}
\usepackage{amssymb}
\usepackage{mathtools}
\usepackage{mathptmx}
\usepackage{natbib}
\usepackage[final,babel]{microtype}
\usepackage[hidelinks]{hyperref}
\usepackage{doi}
\usepackage{siunitx}
\usepackage[margin=1pt]{subcaption}
\usepackage{xcolor}
\PassOptionsToPackage{final}{graphicx}
\usepackage{tikz}
\usetikzlibrary{arrows}
\usetikzlibrary{patterns}
\usepackage{bm}
\usepackage{booktabs}
\usepackage{tabularx}
\usepackage{enumitem}
\usepackage{titlesec}

\title{
\vspace*{-2em}
Monitoring Committee Progress Report \#4\\
\vspace*{1em}
\Large{Numerical Representation of Mountains in Atmospheric Models}}
\author{James Shaw
\vspace{0.5em} \\
\large{Supervisors: Hilary Weller, John Methven, Terry Davies}
\vspace{0.5em} \\
\large{Monitoring Committee: Paul Williams, Maarten Ambaum}}
\date{24th November 2016}

\captionsetup{margin=3pt,font={small}}
%\setlength{\bibsep}{0.3em plus 0.1ex}

\makeatletter
\AtBeginDocument{
  \hypersetup{
    pdftitle = {Monitoring Committee Progress Report \#4},
    pdfauthor = {James Shaw}
  }
}
\makeatother


\newcommand{\TODO}[1]{\textcolor{purple}{TODO: \emph{#1}}}
\begin{document}
\maketitle

\section{Introduction}

An atmospheric model solves the equations of motion in discrete form on a numerical mesh.  This mesh becomes distorted over sloping terrain, and the distortions can lead to larger numerical errors.  With new atmospheric models using increasingly fine mesh spacing, steep slopes become resolved by the mesh.  These steep slopes can result in highly-distorted meshes that can lead to larger numerical errors.  My PhD project seeks out new techniques for reducing numerical errors in regions of steep terrain.

There are several methods for representing terrain on a mesh.  Operational models use some form of terrain-following transformation that distorts horizontal mesh surfaces to accommodate the terrain.  Such meshes are straightforward to represent using a uniform rectangular computational mesh, but sloped terrain distorts the mesh far above the surface, leading to increased errors \citep{schaer2002,klemp2011}.  The cut cell method is an alternative to terrain-following transformations in which a piecewise linear representation of the surface is intersected with a uniform rectangular mesh.  Cut cell meshes are less distorted than terrain-following meshes, but the cut cell method creates arbitrarily small cells that impose severe timestep constraints on explicit numerical methods \citep{klein2009}.  Other representations of terrain have also been used, such as the sloping step method used in the Eta model \cite{mesinger2012} and unstructured tetrahedral meshes \cite{smolarkiewicz-szmelter2011}.

The wide variety of mesh types motivates the development of finite volume methods for arbitrary meshes.  This enables a like-for-like comparison between different types of mesh and allows us to assess the characteristics of new types of mesh.  Further, we hope that the new methods we develop will be less sensitive to mesh distortions than more traditional methods.



\section{Future research}
\label{sec:future}

\section{Personal development}


\bibliographystyle{ametsoc2014}                                                 
\bibliography{references}

\newpage

\section*{Appendix}

\titlespacing\subsection{0pt}{0.8em plus 4pt minus 2pt}{0.2em plus 2pt minus 2pt}

\subsection*{Mathematics modules}
\footnotesize
\begin{tabular}{l l l l}
Spring 2016	& MA3NAT & Numerical Analysis II & unassessed \\
Spring 2015	& MAMNSP & Numerical Solution of Partial Differential Equations  & 78\% \\
\end{tabular}

\subsection*{RRDP modules}
\begin{tabular}{l l}
Summer 2017	& Organising conferences \\
28 Feb 2017	& Getting your first post-doc position \\
9 Nov 2016      & Open Access and research data management \\
24 Mar 2016	& Voice coaching: looking after your voice \\
26--27 Jan 2016 & Preparing to teach (introduction, marking \& feedback, leading small groups) \\
2 Dec 2015	& An essential guide to critical academic writing \\
17 Nov 2015	& Understanding the UK higher education context \\
19 May 2015	& How to avoid plagiarism \\
10 Mar 2015	& How to write a literature review \\
19 Feb 2015	& How to write a paper \\
\end{tabular}

\subsection*{External courses}
\begin{tabular}{l l}
June 2016 & Dynamical core intercomparison project summer school, NCAR \\
13 May 2016 & Peer review: the nuts and bolts, Sense about Science \\
June 2015 & Advanced numerical methods for Earth-system modelling, ECMWF \\
\end{tabular}

\subsection*{Conferences and workshops}
\begin{tabularx}{\linewidth}{l l X}
September 2017 & & SciCADE2017, University of Bath \\
April 2017 & & PDEs on the Sphere, Paris \\
February 2017 & Speaker & Numerical Methods for Geophysical Fluid Dynamics, Imperial College London \\
October 2016 & Speaker & Numerical and computational methods for simulation of all-scale geophysical flows, ECMWF \\
July 2016 & Attendee & 1st GungHo Network meeting, Daresbury Laboratory \\
November 2015 & Attendee & GungHo workshop on next generation weather and climate prediction, UK Met Office \\
June 2015 & Attendee & Hoskins@70 \\
June 2015 & Poster & SCENARIO DTP conference \\
March 2015 & Speaker & Galerkin methods with applications in weather and climate forecasting, ICMS \\
\end{tabularx}

\subsection*{Teaching}
\begin{tabular}{l l l}
Oct 2016 & Teaching assistant & MTMW11 fluid dynamics \\
Oct 2015 & Teaching assistant & MTMG02 atmospheric physics \\
Sep 2015 & Teaching assistant & NCAS summer school \\
Sep 2014 & Course teacher & MPE python and linux short course \\
\end{tabular}

\subsection*{Visits and collaborations}
\begin{tabularx}{\linewidth}{l X}
July 2016 & Organised visit from Simon Clark, stratospheric PhD researcher and YouTube vlogger \\
Summer 2016 & Worked with Hilary's MSc student and dissertation prize winner, Christiana Skea, studying variable timestepping for ODEs \\
June 2016 & Visited NCAR, hosted by Ram Nair \\
2015 -- 2016 & Coauthoring an article about dimensionally-split and multidimensional advection schemes, written with Hilary, her former student Yumeng Chen, and Stephen Pring at the UK Met Office \\
\end{tabularx}


\subsection*{Outreach}
\begin{tabular}{l l l}
14 Jul 2015 & Schools physicist of the year awards \\
14 Jun 2015 & East Reading festival \\
15 Feb 2015 & Brighton science festival \\
\end{tabular}

\subsection*{Presentations}
\begin{tabularx}{\linewidth}{l l X}
17 Nov 2016 & Comp. Atmos. Dyn. group & \\
9 Nov 2016 & PhD group & \\
31 Oct 2016 & HHH group & \\
22 Sep 2016 & PhD poster session & Improving numerical accuracy over steep slopes \\
23 Mar 2016 & Quo Vadis & Numerical representation of orography in dynamical cores (honourable mention) \\
17 Feb 2016 & PhD group & Multidimensional advection schemes for arbitrary meshes \\
9 Feb 2016 & Mesoscale group & Curl-free pressure gradients for accurate modelling of cold air pools \\
19 Oct 2015 & HHH group & Improving modelled mountain flows with alternative representations of terrain \\
27 Apr 2015 & HHH group & A like-for-like comparison between terrain following and cut cell grids \\
21 Apr 2015 & PhD group & Discrete vector calculus on Arakawa C grids \\
12 Feb 2015 & UK Met Office & Poster presentation for Met Office Academic Partnership \\
18 Jan 2015 & PhD group & Python and linux tips \\
17 Dec 2014 & MPECDT jamboree & Poster presentation for Mathematics for Planet Earth Centre for Doctoral Training jamboree \\
12 Sep 2014 & Lunchtime seminar  & Gain control of your documents and code: hands-on with revision control and build automation \\
\end{tabularx}

\end{document}
