\documentclass[a4paper,11pt]{article}
\usepackage{fullpage}
\usepackage{standalone}
\usepackage[utf8]{inputenc}
\usepackage[british]{babel}
\usepackage{csquotes}
\usepackage[T1]{fontenc}
\usepackage{amsmath}
\usepackage{amssymb}
\usepackage{mathtools}
\usepackage{mathptmx}
\usepackage{natbib}
\usepackage[final,babel]{microtype}
\usepackage[colorlinks,linkcolor=blue,citecolor=blue,urlcolor=blue]{hyperref}
\usepackage{doi}
\usepackage{siunitx}
\usepackage[margin=3pt]{subcaption}
\usepackage{xcolor}
\PassOptionsToPackage{final}{graphicx}
\usepackage{tikz}
\usetikzlibrary{arrows}
\usetikzlibrary{patterns}
\usepackage{bm}
\usepackage{booktabs}
\usepackage{tabularx}
\usepackage{enumitem}
\usepackage{titlesec}

\title{
\vspace*{-2em}
Monitoring Committee Progress Report \#5\\
\vspace*{1em}
\Large{Numerical Representation of Mountains in Atmospheric Models}}
\author{James Shaw
\vspace{0.5em} \\
\large{Supervisors: Hilary Weller, John Methven, Terry Davies}
\vspace{0.5em} \\
\large{Monitoring Committee: Paul Williams, Maarten Ambaum}}
\date{May 2017}

\captionsetup{margin=3pt,font={small}}
\setlength{\bibsep}{0.3em plus 0.1ex}

\makeatletter
\AtBeginDocument{
  \hypersetup{
    pdftitle = {Monitoring Committee Progress Report \#5},
    pdfauthor = {James Shaw}
  }
}
\makeatother

\makeatletter
\patchcmd{\ttlh@hang}{\parindent\z@}{\parindent\z@\leavevmode}{}{}
\patchcmd{\ttlh@hang}{\noindent}{}{}{}
\makeatother

\newcommand{\iu}{{i\mkern1mu}}

\defcitealias{openfoam}{{OpenFOAM user guide}}
\begin{document}
\maketitle


\bibliographystyle{ametsoc2014}                                                 
\bibliography{references}

\newpage

\section*{Appendices}

\titlespacing\subsubsection{0pt}{0.8em plus 4pt minus 2pt}{0.2em plus 2pt minus 2pt}

\subsection*{Appendix A: Thesis plan}
The thesis chapters are expected to be as follows:

\subsubsection*{Existing methodologies}
\begin{itemize}[itemsep=0.1em]
	\item Describe Lorenz and Charney--Phillips staggerings for structured quadrilateral meshes
	\item Describe the nonhydrostatic finite volume model with Lorenz staggering for arbitrary meshes \citep{weller-shahrokhi2014}
\end{itemize}

\subsubsection*{A new mesh for representing terrain}
\noindent The slanted cell mesh improves pressure gradient accuracy and avoids severe timestep constraints for explicit methods.
\begin{itemize}[itemsep=0.1em]
	\item Introduce existing types of mesh: terrain-following layers and cut cells
	\item Describe the new slanted cell method
	\item {Two-dimensional test results comparing terrain-following, cut cell and slanted cell meshes
	\begin{itemize}[itemsep=0.1em,topsep=0pt]
		\item Maximum stable timesteps in a prescribed wind field (depends on the mesh geometry only)
		\item A test of a quiescent atmosphere above steep slopes (using the linearUpwind scheme?)
	\end{itemize}}
\end{itemize}

\subsubsection*{A new transport scheme for steep slopes}
\noindent The cubicFit transport scheme is stable and accurate over steep slopes with arbitrary, distorted meshes.
\begin{itemize}[itemsep=0.1em]
	\item Document the cubicFit transport scheme
	\item {Test results comparing linearUpwind and cubicFit transport schemes
	\begin{itemize}[itemsep=0.1em,topsep=0pt]
		\item Two-dimensional transport test over steep slopes, as presented in this report
		\item \citet{lauritzen2012} deformational transport tests on a spherical Earth
		\item \citet{schaer2002} mountain waves test case (should demonstrate that cubicFit is necessary to obtain the reference solution)
	\end{itemize}}
\end{itemize}
	
\subsubsection*{A generalisation of the Charney--Phillips staggering}
\noindent The Charney--Phillips staggering avoids the Lorenz computational mode, but it has only been formulated for structured quadrilateral meshes.  A generalised formulation will be suitable for arbitrary meshes, including the slanted cell mesh.
\begin{itemize}[itemsep=0.1em]
	\item Describe the generalised Charney--Phillips formulation
	\item Describe its implementation into the nonhydrostatic model of \citet{weller-shahrokhi2014}
	\item One or two tests that excite the Lorenz computational mode; compare results between the Lorenz and Charney--Phillips variants of the model
\end{itemize}

\subsubsection*{Possible chapter: grid-scale perturbations and diabatic heating over orography}
\noindent We should demonstrate the efficacy of combining all three work items: the slanted cell mesh, the cubicFit transport scheme, and the generalised Charney--Phillips formulation.  In order to achieve this, we may be able to introduce orography into the two-dimensional idealised simulations of \citet{arakawa-konor1996}.
Results should be compared between Lorenz and Charney--Phillips model variants.  Results could also be compared between mesh types, or results could be compared using the cubicFit and linearUpwind transport schemes.

\newpage

\subsection*{Appendix B: Training record}

\subsubsection*{Mathematics modules}
\footnotesize
\begin{tabular}{l l l l}
Spring 2017	& \href{https://finite-element.github.io}{M5A47}  & Finite elements: numerical analysis and implementation & unassessed \\
Spring 2016	& \href{www.reading.ac.uk/module/document.aspx?modP=MA3NAT&modYR=1516}{MA3NAT} & Numerical Analysis II & unassessed \\
Spring 2015	& \href{www.reading.ac.uk/modules/document.aspx?modP=MAMNSP&modYR=1415}{MAMNSP} & Numerical Solution of Partial Differential Equations  & 78\% \\
\end{tabular}

\subsubsection*{RRDP modules}
\begin{tabular}{l l}
Summer 2017	& Organising conferences \\
28 Feb 2017	& Getting your first post-doc position \\
9 Nov 2016      & Open Access and research data management \\
24 Mar 2016	& Voice coaching: looking after your voice \\
26--27 Jan 2016 & Preparing to teach (introduction, marking \& feedback, leading small groups) \\
2 Dec 2015	& An essential guide to critical academic writing \\
17 Nov 2015	& Understanding the UK higher education context \\
19 May 2015	& How to avoid plagiarism \\
10 Mar 2015	& How to write a literature review \\
19 Feb 2015	& How to write a paper \\
\end{tabular}

\subsubsection*{External courses}
\begin{tabular}{l l}
June 2016 & Dynamical core intercomparison project summer school, NCAR \\
13 May 2016 & Peer review: the nuts and bolts, Sense about Science \\
June 2015 & Advanced numerical methods for Earth-system modelling, ECMWF \\
\end{tabular}

\subsubsection*{Conferences and workshops}
\begin{tabularx}{\linewidth}{l l X}
September 2017 & & \href{https://sites.google.com/site/scicade2017/}{SciCADE}, University of Bath \\
2017 & Speaker & UK Met Office Gung Ho network meeting \\
April 2017 & & \href{https://forge.ipsl.jussieu.fr/heat/wiki/PDEs2017}{PDEs on the Sphere}, Paris \\
March 2017 & Attendee & \href{https://blogs.reading.ac.uk/open-research/open-in-practice-inspirations-strategies-and-methods-for-open-research/}{Open in practice: inspirations, strategies and methods for open research}, University of Reading \\
March 2017 & Participant & \href{https://www.eventbrite.co.uk/e/opendatahack-ecmwf-beyond-weather-explore-creative-uses-of-open-data-tickets-28733656139}{Beyond weather: explore creative uses of open data}, ECMWF hack weekend \\
February 2017 & Speaker & Numerical Methods for Geophysical Fluid Dynamics, Imperial College London \\
January 2017 & Attendee & Research software management, sharing and sustainability, British Library \\
October 2016 & Speaker & \href{http://www.ecmwf.int/en/learning/workshops-and-seminars/workshop-numerical-and-computational-methods-simulation-all-scale-geophysical-flows}{Numerical and computational methods for simulation of all-scale geophysical flows}, ECMWF \\
November 2015 & Attendee & GungHo workshop on next generation weather and climate prediction, UK Met Office \\
June 2015 & Attendee & Hoskins@70 \\
June 2015 & Poster & SCENARIO DTP conference \\
March 2015 & Speaker & \href{http://www.icms.org.uk/workshop.php?id=334}{Galerkin methods with applications in weather and climate forecasting}, ICMS \\
\end{tabularx}

\subsubsection*{Teaching}
\begin{tabular}{l l l}
Oct 2016 & Teaching assistant & MTMW11 fluid dynamics \\
Oct 2015 & Teaching assistant & MTMG02 atmospheric physics \\
Sep 2015 & Teaching assistant & NCAS summer school \\
Sep 2014 & Course teacher & MPE python and linux short course \\
\end{tabular}

\subsubsection*{Visits and collaborations}
\begin{tabularx}{\linewidth}{l X}
July 2016 & Organised visit from \href{https://www.youtube.com/user/SimonOxfPhys}{Simon Clark}, stratospheric PhD researcher and YouTube vlogger \\
Summer 2016 & Worked with Hilary's MSc student, Christiana Skea, studying variable timestepping for ODEs \\
June 2016 & Visited NCAR, hosted by Ram Nair \\
2015 -- 2016 & Coauthoring an article about dimensionally-split and multidimensional transport schemes, written with Hilary, her former student Yumeng Chen, and Stephen Pring at the UK Met Office \\
\end{tabularx}

\subsubsection*{Outreach}
\begin{tabular}{l l l}
14 Jul 2015 & Schools physicist of the year awards \\
14 Jun 2015 & East Reading festival \\
15 Feb 2015 & Brighton science festival \\
\end{tabular}

\subsubsection*{Presentations}
\begin{tabularx}{\linewidth}{l l X}
22--23 Feb 2017 & UK Met Office & Poster presentation for Met Office Academic Partnership \\
17 Nov 2016 & Comp. Atmos. Dyn. group & A review of atmospheric transport schemes \\
9 Nov 2016 & PhD group & Replicable computational atmospheric science \\
31 Oct 2016 & HHH group & Advection over steep slopes \\
22 Sep 2016 & PhD poster session & Improving numerical accuracy over steep slopes \\
23 Mar 2016 & Quo Vadis & Numerical representation of orography in dynamical cores (honourable mention) \\
17 Feb 2016 & PhD group & Multidimensional advection schemes for arbitrary meshes \\
9 Feb 2016 & Mesoscale group & Curl-free pressure gradients for accurate modelling of cold air pools \\
19 Oct 2015 & HHH group & Improving modelled mountain flows with alternative representations of terrain \\
27 Apr 2015 & HHH group & A like-for-like comparison between terrain following and cut cell grids \\
21 Apr 2015 & PhD group & Discrete vector calculus on Arakawa C grids \\
12 Feb 2015 & UK Met Office & Poster presentation for Met Office Academic Partnership \\
18 Jan 2015 & PhD group & Python and linux tips \\
17 Dec 2014 & MPECDT jamboree & Poster presentation for Mathematics for Planet Earth Centre for Doctoral Training jamboree \\
12 Sep 2014 & Lunchtime seminar  & Gain control of your documents and code: hands-on with revision control and build automation \\
\end{tabularx}

\end{document}
