\documentclass[a4paper,11pt]{article}
\usepackage{fullpage}
\usepackage{standalone}
\usepackage[utf8]{inputenc}
\usepackage[british]{babel}
\usepackage{csquotes}
\usepackage[T1]{fontenc}
\usepackage{amsmath}
\usepackage{amssymb}
\usepackage{mathtools}
\usepackage{mathptmx}
\usepackage{natbib}
\usepackage[final,babel]{microtype}
\usepackage[colorlinks,linkcolor=blue,citecolor=blue,urlcolor=blue]{hyperref}
\usepackage{doi}
\usepackage{siunitx}
\usepackage[margin=3pt]{subcaption}
\usepackage{xcolor}
\PassOptionsToPackage{final}{graphicx}
\usepackage{tikz}
\usetikzlibrary{arrows}
\usetikzlibrary{patterns}
\usepackage{bm}
\usepackage{booktabs}
\usepackage{tabularx}
\usepackage{enumitem}
\usepackage{titlesec}
\usepackage{colortbl}

\title{
\vspace*{-2em}
Monitoring Committee Progress Report \#6\\
\vspace*{1em}
\Large{Numerical Representation of Mountains in Atmospheric Models}}
\author{James Shaw
\vspace{0.5em} \\
\large{Supervisors: Hilary Weller, John Methven, Terry Davies}
\vspace{0.5em} \\
\large{Monitoring Committee: Paul Williams, Maarten Ambaum}}
\date{December 2017}

\captionsetup{margin=3pt,font={small}}
\setlength{\bibsep}{0.3em plus 0.1ex}

\makeatletter
\AtBeginDocument{
  \hypersetup{
    pdftitle = {Monitoring Committee Progress Report \#6},
    pdfauthor = {James Shaw}
  }
}
\makeatother

% https://tex.stackexchange.com/q/139401/23339
\graphicspath{{src/mc-report-2017-12/}{build/}}

% https://tex.stackexchange.com/a/79060/23339
\makeatletter
\def\input@path{{src/mc-report-2017-12/}{build/mc-report-2017-12/}{build/}}
\makeatother

\definecolor{done}{rgb}{0.87,0.96,0.87}

\makeatletter
\patchcmd{\ttlh@hang}{\parindent\z@}{\parindent\z@\leavevmode}{}{}
\patchcmd{\ttlh@hang}{\noindent}{}{}{}
\makeatother

\newcommand{\iu}{{i\mkern1mu}}
\newcommand{\TODO}[1]{\textcolor{purple}{TODO: \emph{#1}}}

\defcitealias{openfoam}{{OpenFOAM user guide}}
\begin{document}
\maketitle

\section{Future research}

% Generalisation of Charney--Phillips staggering not going so well
% but high-order transport scheme going much better

\section{Personal development}

I continue to give regular talks at group meetings and I have arranged to give a lunchtime seminar in January 2018.
I have also been invited by \href{http://staff.southwales.ac.uk/users/8005-jkent}{James Kent} to speak at a mathematics departmental seminar at University of South Wales where Dr Kent lectures.

I also continue searching for postdoctoral vacancies.
I was unsuccessful in my application for a Mozilla science fellowship, though I was very happy to be placed amongst the top 5\% of more than 1000 applicants.
I was offered a position as a scientific software engineer at the Met Office, but chose to decline since the role was focused on software development with few opportunities for scientific research.
I was also interviewed for a postdoctoral position developing numerical methods for flood forecasts at the University of Sheffield, but the position was given to a more experienced candidate.

I have worked with David Dritschel at the University of St Andrews and Alan Blyth, Doug Parker and Steven B\"{o}ing at the University of Leeds to submit an EPSRC proposal that extends the Moist Parcel-in-a-Cell model \citep{boeing2017} for studying cloud turbulence.  We expect the EPSRC to reach a decision around March 2018.
I am also a named postdoctoral researcher on a NERC proposal to study physics-dynamics coupling that Hilary submitted in collaboration with Jemma Shipton, Colin Cotter, Peter Clark and John Thuburn in July 2017.

\nocite{*}

\bibliographystyle{ametsoc2014}                                                 
\bibliography{src/mc-report-2017-12/references}

\newpage

\titlespacing\subsubsection{0pt}{0.8em plus 4pt minus 2pt}{0.2em plus 2pt minus 2pt}

\section*{Appendix A: Thesis plan}
\footnotesize
A work-in-progress draft is available at \url{http://www.datumedge.co.uk/publications/phd-thesis.pdf}.

\subsubsection*{1. Introduction}
\begin{tabularx}{\linewidth}{>{\hsize=0.9in}X X}
Not started & This project is motivated by the need for alternative horizontal and vertical representations of Earth's atmosphere in the proximity to mountainous terrain
\end{tabularx}

\subsubsection*{2. Numerically stable transport over steep slopes}
\noindent The cubicFit transport scheme is stable and accurate over steep slopes with arbitrary, distorted meshes.
\vspace*{0.5em}

\begin{tabularx}{\linewidth}{>{\hsize=0.9in}X X}
\rowcolor{done}	Complete & Review of existing transport schemes and motivation for the cubicFit transport scheme \\
\rowcolor{done}	Complete & Document the cubicFit transport scheme \\
\addlinespace[0.5em]
	 & Test results comparing a standard linear upwind scheme and the cubicFit transport scheme: \\
\rowcolor{done}	Complete & \quad\textbullet\enspace horizontal transport test above steep slopes \\
\rowcolor{done}	Complete & \quad\textbullet\enspace terrain-following transport test above steep slopes \\
	Mostly complete & \quad\textbullet\enspace deformational transport tests on a spherical Earth \\
\end{tabularx}

\subsubsection*{3. A new mesh for representing the atmosphere above terrain}
\noindent The slanted cell mesh improves pressure gradient accuracy and avoids severe time-step constraints.
\vspace*{0.5em}

\begin{tabularx}{\linewidth}{>{\hsize=0.9in}X X}
\rowcolor{done} Complete & Introduction to pressure gradient errors and existing methods to improve pressure gradient accuracy \\
\rowcolor{done} Complete & Describe the new slanted cell method \\
\rowcolor{done} Complete & Transport test over a mountainous lower boundary using terrain-following, cut cell and slanted cell meshes \\
\rowcolor{done} Complete & A two-dimensional test of a quiescent atmosphere above steep slopes, comparing terrain-following, cut cell and slanted cell meshes \\
\end{tabularx}


\newpage

\section*{Appendix B: Training record}

\subsubsection*{Mathematics modules}
\begin{tabular}{l l l l}
Spring 2017	& \href{https://finite-element.github.io}{M5A47}  & Finite elements: numerical analysis and implementation & unassessed, partially completed \\
Spring 2016	& \href{www.reading.ac.uk/module/document.aspx?modP=MA3NAT&modYR=1516}{MA3NAT} & Numerical Analysis II & unassessed \\
Spring 2015	& \href{www.reading.ac.uk/modules/document.aspx?modP=MAMNSP&modYR=1415}{MAMNSP} & Numerical Solution of Partial Differential Equations  & 78\% \\
\end{tabular}

\subsubsection*{RRDP modules}
I have completed the requisite 11 RRDP modules.

\vspace*{0.5em}
\begin{tabular}{l l}
23 June 2017    & Graduate school conference \\
3 May 2017	& Effective CVs \\
28 Feb 2017	& Getting your first post-doc position \\
9 Nov 2016      & Open Access and research data management \\
24 Mar 2016	& Voice coaching: looking after your voice \\
26--27 Jan 2016 & Preparing to teach (introduction, marking \& feedback, leading small groups) \\
2 Dec 2015	& An essential guide to critical academic writing \\
17 Nov 2015	& Understanding the UK higher education context \\
19 May 2015	& How to avoid plagiarism \\
10 Mar 2015	& How to write a literature review \\
19 Feb 2015	& How to write a paper \\
\end{tabular}

\subsubsection*{External courses}
\begin{tabular}{l l}
June 2016 & Dynamical core intercomparison project summer school, NCAR \\
13 May 2016 & Peer review: the nuts and bolts, Sense about Science \\
June 2015 & Advanced numerical methods for Earth-system modelling, ECMWF \\
\end{tabular}

\subsubsection*{Conferences and workshops}
\begin{tabularx}{\linewidth}{l l X}
September 2017 & Speaker & \href{https://sites.google.com/site/scicade2017/}{International conference on scientific computation and differential equations}, University of Bath \\
August 2017 & Co-organiser & \href{https://frontiers2017.wordpress.com/}{Frontiers in natural environment research}, Imperial College London \\
July 2017 & Speaker & UK Met Office GungHo network meeting, University of Exeter \\
June 2017 & Participant & \href{https://www.software.ac.uk/c4rr}{Docker containers for reproducible research}, University of Cambridge \\
April 2017 & Speaker & \href{https://forge.ipsl.jussieu.fr/heat/wiki/PDEs2017}{PDEs on the Sphere}, École normale supérieure, Paris \\
March 2017 & Attendee & \href{https://blogs.reading.ac.uk/open-research/open-in-practice-inspirations-strategies-and-methods-for-open-research/}{Open in practice: inspirations, strategies and methods for open research}, University of Reading \\
March 2017 & Participant & \href{http://www.effective-quadratures.org/eq2017}{Effective quadratures workshop}, University of Cambridge \\
February 2017 & Invited speaker & Numerical methods for geophysical fluid dynamics, Imperial College London \\
January 2017 & Attendee & Research software management, sharing and sustainability, British Library \\
December 2016 & Invited speaker & \href{https://www.rmets.org/events/meteorological-research-within-university-reading-2016}{South-East local centre meeting, Royal Meteorological Society} \\
October 2016 & Speaker & \href{http://www.ecmwf.int/en/learning/workshops-and-seminars/workshop-numerical-and-computational-methods-simulation-all-scale-geophysical-flows}{Numerical and computational methods for simulation of all-scale geophysical flows}, ECMWF \\
November 2015 & Attendee & GungHo workshop on next generation weather and climate prediction, UK Met Office \\
June 2015 & Attendee & Hoskins@70 \\
June 2015 & Poster & SCENARIO DTP conference \\
March 2015 & Speaker & \href{http://www.icms.org.uk/workshop.php?id=334}{Galerkin methods with applications in weather and climate forecasting}, ICMS \\
\end{tabularx}

\subsubsection*{Teaching}
\begin{tabular}{l l l}
Oct 2016 & Teaching assistant & MTMW11 fluid dynamics \\
Oct 2015 & Teaching assistant & MTMG02 atmospheric physics \\
Sep 2015 & Teaching assistant & NCAS summer school \\
Sep 2014 & Course teacher & MPE python and linux short course \\
\end{tabular}

\subsubsection*{Visits and collaborations}
\begin{tabularx}{\linewidth}{l X}
July 2016 & Organised visit from \href{https://www.youtube.com/user/SimonOxfPhys}{Simon Clark}, stratospheric PhD researcher and YouTube vlogger \\
Summer 2016 & Worked with Hilary's MSc student, Christiana Skea, studying variable timestepping for ODEs \\
June 2016 & Visited NCAR, hosted by \href{http://www.image.ucar.edu/staff/rnair/}{Ram Nair} \\
2015 -- 2017 & Coauthoring an article about dimensionally-split and multidimensional transport schemes, written with Hilary, her former student \href{https://www.clisap.de/research/a:-climate-dynamics-and-variability/crg-numerical-methods-in-geosciences/team-members/yumeng-chen/}{Yumeng Chen}, and Stephen Pring at the UK Met Office \\
\end{tabularx}

\subsubsection*{Outreach}
\begin{tabular}{l l l}
17 Mar 2017 & ``\href{https://thesocialmetwork.wordpress.com/2017/03/17/simulating-wind-on-computers/}{The advection process: simulating wind on computers}'', Social Metwork blog article \\
14 Jul 2015 & Schools physicist of the year awards \\
14 Jun 2015 & East Reading festival \\
15 Feb 2015 & Brighton science festival \\
\end{tabular}

\subsubsection*{Presentations}
\begin{tabularx}{\linewidth}{l l X}
2 Feb 2018 & University of South Wales \\
22 Jan 2018 & HHH group \\
16 Jan 2018 & Lunchtime seminar \\
26 Oct 2017 & Comp. Atmos. Dyn. group & High-order finite volume advection \\
17 Oct 2017 & Mesoscale group & Numerical advection of chemical species \\
27 Sep 2017 & DARC group & Subsampling for uncertainty quantification \\
6 Jul 2017 & Comp. Atmos. Dyn. group & Charney--Phillips review and automated testing for OpenFOAM applications \\
19 Jun 2017 & HHH group & Subsampling for uncertainty quantification \\
17 Nov 2016 & Comp. Atmos. Dyn. group & A review of atmospheric transport schemes \\
9 Nov 2016 & PhD group & Replicable computational atmospheric science \\
31 Oct 2016 & HHH group & Advection over steep slopes \\
22 Sep 2016 & PhD poster session & Improving numerical accuracy over steep slopes \\
23 Mar 2016 & Quo Vadis & Numerical representation of orography in dynamical cores (honourable mention) \\
17 Feb 2016 & PhD group & Multidimensional advection schemes for arbitrary meshes \\
9 Feb 2016 & Mesoscale group & Curl-free pressure gradients for accurate modelling of cold air pools \\
19 Oct 2015 & HHH group & Improving modelled mountain flows with alternative representations of terrain \\
27 Apr 2015 & HHH group & A like-for-like comparison between terrain following and cut cell grids \\
21 Apr 2015 & PhD group & Discrete vector calculus on Arakawa C grids \\
12 Feb 2015 & UK Met Office & Poster presentation for Met Office Academic Partnership \\
18 Jan 2015 & PhD group & Python and linux tips \\
17 Dec 2014 & MPECDT jamboree & Poster presentation for Mathematics for Planet Earth Centre for Doctoral Training jamboree \\
12 Sep 2014 & Lunchtime seminar  & Gain control of your documents and code: hands-on with revision control and build automation \\
\end{tabularx}

\section*{Appendix C: Publication milestones}

\begin{tabular}{l l}
10 Jun 2015 & First MWR manuscript submitted \\
19 Aug 2015 & Major revisions required to MWR manuscript \\
29 Oct 2015 & Second MWR manuscript submitted\\
9 Dec 2015 & Major revisions required to MWR manuscript \\
5 Feb 2016 & Third MWR manuscript submitted
	\vspace*{1em} \\
2 Feb 2017 & First JCP manuscript submitted \\
13 Mar 2017 & Minor revisions required to JCP manuscript \\
21 Apr 2017 & Second JCP manuscript submitted \\
\end{tabular}

\end{document}
